\documentclass[12pt,draftclsnofoot,journal,onecolumn]{IEEEtran}
\usepackage{amsfonts}
\usepackage{amsmath,amssymb}
\usepackage{graphicx}
\usepackage{color, soul}
\usepackage{algorithm,algorithmic}
\usepackage{bm}
\usepackage{booktabs}
\usepackage{flushend}
\usepackage{tikz}
\usepackage{cite}
\usetikzlibrary{arrows}
\usepackage{subfigure}
\usepackage{indentfirst}
\usepackage[amsmath,thmmarks]{ntheorem}
\usepackage{theorem}
\usepackage{acronym}  % make an acronym
\usepackage{balance}
\newtheorem{theorem}{\bf Theorem}
\newtheorem{proposition}{\bf Proposition}
\newtheorem{lemma}{\bf Lemma}
\newtheorem{definition}{Definition}
\newtheorem{remark}{\bf Remark}

\theoremheaderfont{~~~\it}\theorembodyfont{\upshape}%
\theoremstyle{nonumberplain}
\theoremseparator{}
\theoremsymbol{\rule{1ex}{1ex}}
\newtheorem{proof}{Proof:}
\acrodef{EAR}[EAR]{element activation ratio}
\acrodef{SNR}[SNR]{signal-to-noise ratio}
\acrodef{TC}[TC]{transmission coefficients}
\acrodef{US-RIS}[US-RIS]{user-side RIS}
\acrodef{BSS-RIS}[BSS-RIS]{base-station-side RIS}
\acrodef{DoF}[DoF]{degree of freedom}
\acrodef{FPGA}[FPGA]{field programmable gate array}
\acrodef{RF}[RF]{radio-frequency}
\acrodef{RIS}[RIS]{reconfigurable intelligent surfaces}
\acrodef{UE}[UE]{user equipment}
\acrodef{DL}[DL]{downlink}
\acrodef{TA}[TA]{transmit antenna}
\acrodef{RA}[RA]{receive antenna}
\acrodef{LoS}[LoS]{line-of-sight}
\acrodef{UL-TBF}[UL-TBF]{uplink transmit beamforming}
\acrodef{TPS}[TPS]{transmit phase shifter}
\acrodef{RC}[RC]{receiver combining}
\acrodef{AWGN}[AWGN]{additive white gaussian noise}

\def \H {^H}
\def \opt {^{\text{opt}}}
\def \v {\bm v}
\def \w {\bm w}
\def \g {\bm g}
\def \f {\bm f}
\def \T {\bm \Theta}
\def \t {\bm \theta}
\def \x {\bm \xi}
\def \Pmax {P_{\text{max}}}
\def \ml {multi-layer }
\def \tb {transmit beamformer }
\def \sl {single-layer }
\newcommand{\RNum}[1]{\uppercase\expandafter{\romannumeral #1\relax}}
\renewcommand{\algorithmicrequire}{\textbf{Input:}}   %Use Input in the format of Algorithm
\renewcommand{\algorithmicensure}{\textbf{Output:}}  %UseOutput in the format of Algorithm
\ifCLASSINFOpdf
\else
\fi
\hyphenation{op-tical net-works semi-conduc-tor}
\begin{document}
\title{Sensing RIS}
\author{{Authors
\vspace*{-1em}}
\thanks{Beijing National Research Center for Information Science and Technology (BNRist)}
%\thanks{Corresponding author: Linglong Dai.}
}

\maketitle

\begin{abstract}
xxx
\end{abstract}

\begin{IEEEkeywords}
xxx
\end{IEEEkeywords}

\section{System Model}
\label{System Model}
\subsection{MISO case}
\label{MISO case}

Signal model:
\begin{equation}
\label{Signal model}
y=\bm f^{H}\bm \Theta\bm G\bm ws+z,
\end{equation}
where $\bm f\in \mathbb C ^{N\times 1}$ and $\bm G \in \mathbb C^{N\times M}$ denote the channel spanning from the RIS to the user and the channel spanning from the BS to the RIS, respectively; $\bm w\in \mathbb C^{M\times 1}$ denotes the beamforming vector at the transmitter BS; $s$ denotes the transmitted normalized symbol; $z$ denotes the \ac{AWGN} introduced at the receiver user.

Interference field at the $n$-th RIS element:
\begin{equation}
\label{interference}
E_{n}=
\end{equation}


\appendices
\section{xxx}

\section*{Acknowledgments}


\footnotesize
\balance 
\bibliographystyle{IEEEtran}
\bibliography{SensingRIS, IEEEabrv}

\end{document}











